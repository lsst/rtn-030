\setActionItemPrefix{S-}


\section{Introduction}
This is the Rubin information and data security plan for operations.
This renders \citeds{LDM-324} obsolete.

This security plan conforms with the construction era security documentation:
\begin{itemize}
\item “LSST Information Classification Policy” \citedsp{LPM-122}
\item “LSST Master Information Security Policy” \citedsp{LPM-121}
\end{itemize}

Note has been taken of \cite{TCISSG}.

During the ramp up to operations further security requirements were given to the project by the agencies.
The response to these is in \gls{DMTN}-199.

The \gls{RDM} department is concerned with the operations and maintenance  of Rubin constructed software, hardware and networks.
Development remains open-source in nature.

Data from the telescope on Cerro Pach\'{o}n is transmitted over Rubin controlled networks to \gls{SLAC} in  Menlo Park USA.
After an agreed embargo period all raw data will be transmitted to the French Data Facility at the computing center for \gls{IN2P3} in Lyon, France, and at least some raw data is transmitted to the United Kingdom Data Facility operating on the IRIS infrastructure and in particular utilizing facilities at \gls{RAL} and Lancaster University.
Processing occurs in all locations using a coordinated release of the Rubin \gls{Science Pipelines}.

Access to data will be provided to authorized  users via a US \gls{DAC}.
A data access center similar to the \gls{US} \gls{DAC} will also be provided in Chile.

For data taken initially we have been requested to comply with \citeds{NIST.SP.800-171}; the compliance matrix is provided in  \citeds{RTN-082}.
SLAC is subject to \citeds{nist800-53}


This document does not do a detailed threat analysis though this should be and is done for individual systems and Zones. \cite{TCISSG} is a good guide for software threats.

\section{Information types in Rubin Observatory} \label{sec:infotypes}

\VRO is an open source project with no confidentiality requirements on the software.
The software project’s integrity requirements are met through the combination of
processes and controls which provide verified user access and protected credentials.
The majority of software testing is conducted using open simulated and observational data sets.

All data, after the embargo period,  is classified as Internal for two years before it becomes public.
Internal here means available to data rights holders as per \citeds{RDO-013}.
Data rights holders are instructed not to disseminate data outside of the collaboration.
This is the customary protection for this data in the field of optical astronomy.

\subsection{Presence of controlled information}\label{sec:cui}
Rubin observatory data is not considered \gls{CUI}, however we have been asked to embargo data for eighty hours and up to ten days for some images.

\subsection{Controlled Catalog}\label{sec:controllescat}
To avoid alerting on certain man made objects Rubin has access to a catalog which is considered \gls{OUO}.
This catalog is held only in a secure enclave within S3DF.
All access to it is via queries into that system for specific times and positions.

\subsection{Embargoed Data} \label{sec:embargo}
While pixel data is embargoed it is kept on encrypted disks within the embargo rack in S3DF.
This is in a locked rack with limited access and observed by cameras.

\subsection{Unembargoed Data}
Once the pixel data reaches its embargo age it moves to the main S3Df enclave where it is no longer encrypted and is available to data rights holders. It is still not public but is no longer of a sensitive nature.


\section{Cyber incident response}
In this document many threat vectors are identified.
Rubin and SLAC  IT, Chile DevOps Team and many other \gls{DM} staff are highly aware of these threat vectors.
If an incident is observed we:
\begin{itemize}
\item Notify the \gls{ISO} \citedsp{LPM-121}
\item \gls{ISO} coordinates the security team analysis to determine the scope of the incident and establish it is a real threat
\item \gls{ISO} notifies \gls{PMO} (project manager and deputies )
\item \gls{ISO} notifies SLAC \gls{CISO}
\item The \gls{ISO} decides which information and  how widely to broadcast to all staff.
\item Investigation and remediation are initiated - which may include disaster recovery.
\item The combined Rubin \gls{IT} team (SLAC, and AURA)  take action to isolate, remove or  switch off affected systems.
\item Staff are informed of the plan to return systems to service.
\item \gls{ISO} organizes a Post Incident Review
\end{itemize}


\section{System description} \label{sec:desc}

\VRO produces around 20TB of astronomical data per night for the 10 year \gls{LSST}
The control of the the observatory is part of Telescope and site and is covered in \secref{sec:tsarc}.
Data are processed in \gls{SLAC}, \gls{IN2P3} and \gls{ROE}.
A publicly available alert stream emanates from the \gls{USDF} at \gls{SLAC}.
Processing is the responsibility's of \gls{RDP} department while \gls{Quality Assurance} is carried out under \gls{RPF} department.

The Rubin \gls{Operations} Plan \citeds{RDO-018} gives more details.
The \gls{USDF} specification is in \citeds{DMTN-189} but some architecture details are provided in \secref{sec:dparc}.






\subsection{Data verification and quality assurance }
Members of the \gls{RPF} department assess the data quality at short and long timescales.
They require access to tools installed at the summit as well as in the data facilities.
In particular they will need access to images as they are processed to generate alerts.

These team members will have to have \gls{SLAC} accounts to perform this work - \gls{SLAC} accounts are governed by SLAC procedures \cite{SLACOB} and \cite{SLACNH}.
In addition they will need summit accounts as governed by \citeds{ittn-045} and \citeds{ittn-010}.



\subsection{Data processing}
\subsubsection{Prompt  Processing}
Prompt processing is performed on embargoed data with in SLAC.
This data is subject to \citep{ACP}.

\subsubsection{Data Release Processing}
DRP is performed only on unembargoed data which is fully available to Rubin data rights holders.
Processing is carried out at three sites.
Each site has its own security policies:

\begin{itemize}
\item SLAC lists cyber policies on \url{https://it.slac.stanford.edu/cybersecurity/compliance}.
\item UKDF falls under the \gls{IRIS} security policies listed on \url{https://www.iris.ac.uk/security/}.
\item FrDF lists cyber policies and compliance on \url{https://doc.lsst.eu/cybersecurity/cybersecurity.html}.
\end{itemize}



\subsection{Data access}
After a short embargo period image and derived data are made available to all
data rights holders as defined in \citeds{LDO-13}.

Front end data access via the \gls{RSP} will be hosted on a \gls{cloud} provider such as Google.
Thus not requiring community science users to have \gls{SLAC} accounts.

There will  be a \gls{DAC} in Chile to support Chilean users.
The \gls{UK} intends to host a DAC for \gls{UK} users.

In addition there will be a set of \gls{IDAC}s which will usually serve a portion of the data e.g. perhaps only catalogs or only the object catalog.


\subsection{Rubin \gls{Director}'s Office}
Rubin headquarters are in Tucson Arizona where a few services are also deployed such as Jira and Confluence.
These are discussed more in \secref{sec:rdo}


%ENclave text from KT  https://docs.google.com/document/d/1apNSWtIpbS7aCitaF_K0JXLAAY9PvCEmmvnNvHmecdk/edit#heading=h.nwgy0freb3zv


\section{Data Management system architecture} \label{sec:dparc}
The overall system architecture is available in \citeds{LDM-148}.
Details on the \gls{USDF} specifications are given in \citeds{DMTN-189}.

\secref{sec:desc} gives a high level overview of the system. Architecturally we look
at this as a set of enclaves.
As images are processed in the Prompt and Offline Production enclaves, their resulting data products are stored in the \gls{Archive} enclave and made available to the \gls{DAC} enclave where data rights holders can access and analyze them.
In addition, Rubin Observatory staff will use the Development/Integration enclave to maintain the Observatory's \gls{software} tools and systems and to develop new versions of them.

These enclaves are further described here and for each a series of subsections explore :


\begin{enumerate}
\item Threats and Security infrastructure
\item Disaster recovery
\end{enumerate}

\subsection{US Data Facility}\label{sec:usdf}

A number of enclaves are within the \gls{USDF}.
Responsibility for the design, operations, maintenance, and security of these systems is held by the \gls{USDF} Lead at SLAC, who may delegate certain functions to the \gls{USDF} Deputy Lead, \gls{USDF} Technical Lead, other \gls{USDF} staff, S3DF staff, and SLAC IT staff.

In particular, at present the USDF Technical Lead serves as the Rubin Information Systems Security Manager for all USDF systems.

\subsubsection{Prompt \gls{Enclave}} \label{sec:promptenc}

The Prompt enclave holds and processes pixel data that is subject to embargo, meaning that it is to be accessed only by Rubin and \gls{Commissioning} staff, not by the Rubin data rights community.

It receives images from the Observatory facilities in Chile via a Long Haul Network connection.
It stores these and processes them into Prompt data products of three main types:
alerts for things that have moved or changed, measurement catalogs, and processed images.
The alerts can be further subdivided into \emph{streak} alerts for objects that have moved a long distance and \emph{non-streak} alerts for all other objects.
Measurements in the catalogs follow the same subdivision. Images may be \gls{Commissioning} images used for testing and characterizing the Observatory systems, normal science images without significant \emph{streaks}, or delayed science images that do contain significant \emph{streaks}.

\emph{Streak} alerts corresponding to satellites in the \gls{OUO} catalog at \gls{SLAC} will not be released.
Uncatalogued \emph{Streak} alerts agreed with the \gls{JPL} \gls{NEO} group  are to be published at least to \gls{MPC}.
\emph{Non-streak} alerts are to be published to the world at large within 60 seconds of the original raw image being taken.
Normal science images are made available to data rights holders in the \gls{DAC} after an 80 hour embargo period.
Delayed science images, as identified by the agencies, are released after their specific embargo period. \gls{Commissioning} images are made available to data rights holders after a 30 day embargo period.

All Prompt data products are checked for quality by automated systems but also by human operators from the Rubin Observatory staff, who have access to all images and data products in order to perform spot checks or follow ups.

\paragraph{ Threats and Security infrastructure}
The obvious threat surfaces here are :
\begin{enumerate}
\item Transmission of Data from Chile. IPSec built into the routers will be used on the \gls{LHN}. \citeds{DMTN-108} discusses threats in this realm a little more.
\item Transmission to \gls{JPL}.  This transmission will only include measurements of potential NEOs, not pixel data, and will be over internet using \gls{TLS}.
\item Pipeline access for Prompt Processing. The pipelines will run on resources dedicated to the Prompt \gls{Enclave} and without direct external Internet connections for the duration of the processing, mitigating external, intra-node, and side channel attacks.
\item Staff access for \gls{QA}. All the usual user threats such as phishing apply - these users are however governed bu \gls{SLAC} security policies\footnote{\url{https://it.slac.stanford.edu/cybersecurity/compliance}}.
\item QA tools. The web accessible QA tools should have a threat analysis performed by SLAC or our Security consultants although they will be behind SLAC authentication (including 2FA), Rubin authorization (via Gafaelfawr \citeds{DMTN-234}) and the S3DF HTTPS Ingress. These tools are also only accessible by staff and probably pose a low risk.
\item Physical access to storage.  For the initial 80 hours (30 days in commissioning) after acquisition the data is maintain on encrypted storage with in the physically locked Embargo rack.
\end{enumerate}

\paragraph{Disaster recovery}
All embargoed raw data is also stored on a secure server in Chile as part of the Observatory \gls{Operations} Data Service (OODS), hence it can be retransmitted as needed.
For a limited time greater than 80 hours, embargoed raw data is also available from the LSSTCam \gls{Camera} Control System, giving a third data location during the Operations embargo period.
Embargoed data products can be regenerated from the raw data.
In the case of a total wipe out of the Prompt Enclave systems, use of Chef, \gls{Kubernetes}, etc. allow rapid redeployment.
See also the \gls{USDF} Disaster Recovery Plan \citeds{RTN-078}.

\subsubsection{Satellite Catalog Sub-\gls{Enclave}} \label{sec:satcat}
This enclave holds the only \gls{OUO} data on the project, a catalog of satellite orbit information.
This catalog will be held on two dedicated servers (for availability) within the Embargo Rack.
Access to these servers is limited to a highly vetted list of administrators and operators, including the \gls{USDF} Lead, \gls{USDF} Deputy Lead, and \gls{USDF} Technical Lead, who are a (very) small subset of Rubin staff.

\paragraph{ Threats and Security infrastructure}
\begin{enumerate}
\item Catalog retrieval.  The catalog is downloaded from an external source via HTTPS.  Connections originate in the Sub-Enclave from a dedicated server and are authenticated and authorized via a service account provided by the external source.  This minimizes access to the catalog in transit and avoids intra-node and side channel attacks on the catalog service.
\item Catalog lookup.  The catalog is held in the memory of a service that provides match/no match results for each streak measurement in a list.  The service is only accessible by S3DF systems, not the Internet.  This minimizes attacks via the service \gls{API} and prevents attacks via storage.
\end{enumerate}

\paragraph{Disaster recovery}
Since the catalog is downloaded at startup and periodically, it will be refreshed automatically after any disaster.

\subsubsection{ Offline Production  \gls{Enclave}}

The Offline Production enclave holds data that is not embargoed but is not yet released to the data rights community.
Access to this data is limited to Rubin staff.

Each year (or more frequently), the Offline Production enclave takes the raw images accumulated to date in the \gls{Archive} and reprocesses them to generate highly accurate, consistent images and measurement catalogs, known as a Data Release. These data products are stored in the \gls{Archive} and made available to data rights holders in the DAC after they have been checked by automated systems and after Rubin Observatory staff has vetted, characterized, and documented them. Offline Production is split between the USDF and the \gls{FrDF} and UKDF. Each Data Facility performs part of the computations and exchanges its results with the others, so all have a complete set of data products at release time.


\paragraph{ Threats and Security infrastructure}
\begin{itemize}
\item Offline production data is no longer embargoed ergo not considered under threat.  Early release of data to the data rights community or even release of small amounts of data to the world at large is more of a ``science'' problem, less a ``security'' problem.
\item Although data exchange among the facilities uses encryption (secure HTTP over \gls{TLS}), if data were to be intercepted in transfer between sites, this could only occur after the embargo period hence the security risk is low.
\item Malicious users could disrupt data or processing.
We are using standard tooling from \gls{HEP} which has been in use for many years and gives a level confidence of their suitability in this scientific endeavor.
Still internal users remain a major risk - we maintain an inclusive project and try to avoid disgruntled team members.
\end{itemize}
\paragraph{Disaster recovery}
The Offline Production systems run in batch and \gls{Kubernetes} and can be reconstituted after a disaster.
The data being processed can be regenerated from the \gls{Archive} Enclave below.


\subsubsection{ \gls{Archive}  Enclave}
The raw images, released data products, and other records of the survey such as commands, events, and telemetry from Observatory systems are all stored in the \gls{Archive}.
As the permanent scientific record of the survey, no more than 1\% of the raw images or telemetry may be lost or corrupted according to Rubin requirements.

To help ensure this, the French Data Facility maintains a disaster recovery copy of all raw images and selected data products. Additional copies of some raw images and data products will be stored in Observatory systems in Chile.  Raw images and key data products are also stored on tape backup at \gls{SLAC}.

\paragraph{ Threats and Security infrastructure}

\begin{itemize}
\item \gls{Archive} data is no longer embargoed ergo not considered under threat.  All Archive data becomes public after the two year proprietary period, so any disclosure of small portions to the world at large is a premature release and not really a security issue.
\end{itemize}
\paragraph{Disaster recovery}
Post embargo \gls{FrDF} keeps a full copy of the raw data.

\subsubsection{Development and Integration  \gls{Enclave}}
Rubin Observatory and \gls{USDF} staff will use this enclave to build and test new versions of \gls{software} and services to be deployed in the other enclaves.

\paragraph{Threats and Security infrastructure}

\begin{itemize}
\item Developers have a higher level of access than data rights holders.
This is a necessary and accepted risk.
\item All developers must have SLAC accounts and therefore adhere to SLAC access rules e.g. \gls{FACTS} checking etc.
\end{itemize}

\paragraph{Disaster recovery}
SLAC keep tape backups.

All code is deployed using \gls{Kubernetes} or Chef and hence fairly easily recoverable in case of catastrophic failure.


\subsection{ \gls{USDF} \gls{DAC} Enclave}
The USDF \gls{DAC} is hosted on Google and  is the responsibility of SLAC.
All deployments on the USDF DAC are made by \gls{SQuaRE} using Phalanx.

Data rights holders will use the services and systems in this enclave to work with the survey data products.
It is therefore a general-purpose scientific computing facility. Generally users will interact with the Rubin \gls{Science Platform} (\gls{RSP}), which is composed of a web-based Portal Aspect providing a guided user interface for accessing and analyzing the data, a Notebook Aspect providing an interactive, flexible, programming-oriented interface, and an API Aspect providing an programmable access service.
Users of the DAC may connect from anywhere in the world over the Internet; all such users will be authenticated and authorized before accessing any \gls{RSP} service.
The \gls{RSP} is hosted on a \gls{cloud} service, currently  Google Cloud Platform.

The DAC retrieves the released data products from the \gls{Archive} \gls{Enclave} via protocols and services authenticated at a service account level only. While end-user identities may be included for audit and accounting purposes, fundamentally the DAC exists to provide access to all \gls{Archive} contents.

\subsubsection{ Threats and Security infrastructure}
The \gls{RSP} is  an attractive generic target due to its computing resources.
There is some user generated data which is mildly sensitive.
Hosting it on a cloud provider reduces risk considerably for the \gls{Archive} enclave, and also leverages the security products and services made available by the hosting provider.
\citeds{SQR-041} provides a risk assessment for the \gls{RSP}.
\citeds{DMTN-193} provides a more in depth web risk analysis.

\begin{itemize}
\item We will have a lot of users which could be problematic. Keeping the data rights holders on the \gls{cloud} allows a clean separation of concerns between SLAC for processing and archive and the more public facing \gls{RSP}.

\item Backend archive services could provide another attack surface.  These are governed by \gls{SLAC} security.
\end{itemize}

\subsubsection{Disaster recovery}
For the software and deployed systems, all information needed to reconstitute the US DAC is stored in public repositories of container images and configuration files.

For user  spaces we rely on \gls{cloud} provider redundancy/backup/recovery.

The data in the cloud is merely a cache; a full copy is always held at the \gls{USDF} hence any Rubin data at the \gls{DAC} is expendable.

Further considerations are covered in \citeds{rtn-059}.


\subsection{ Chile \gls{DAC} Enclave}
This proposed \gls{DAC} in Chile is covered in \cite{LDM-572}.
We will  start work on this in 2025 nearer the start of operations.
Chile DevOps team are responsible for the Chile \gls{DAC}.
Most applications deployed on the Chile DAC will be deployed by DM's \gls{SQuaRE} team.
The applications will be same as deployed on the USDF \gls{DAC}.

\subsubsection{ Threats and Security infrastructure}

\begin{itemize}
\item The Chile DAC is within the Recinto data center and covered by \gls{AURA}/COS security measures.\item All Rubin traffic is run through a security appliance (currently Zeek).
\item Selected Chilean users have access to the \gls{DAC}. We will keep the \gls{DAC} and the users confined with least privileges. We will use a caching mechanism analogous to the Cloud \gls{DAC} system to restrict access to the object store for the external users.
\item All access will be via \gls{RSP} pods and hence containerized - escalation potential from in side the container will be carefully monitored.
\end{itemize}
\subsubsection{Disaster recovery}
The Chile disaster recovery plan will cover the Chile \gls{DAC} \citeds{ittn-055}.



\subsection{FrDF Processing  \gls{Enclave}}
40\% of \gls{DRP} will be done at \gls{IN2P3}.
A full back up of the raw data will also be held there.
The \gls{IN2P3} computing infrastructure is described in \url{https://doc.lsst.eu/}.
The \gls{CNRS} staff at \gls{IN2P3} are fully responsible fir the \gls{FrDF}.

\subsubsection{Threats and Security infrastructure}
The considerations for this Enclave are a combination of those for the Offline Production Enclave and \gls{Archive} Enclave.
Access will be granted only to \gls{IN2P3} staff and Rubin staff.
IN2P3 have their own cyber security procedures which will be adhered to.
\subsubsection{Disaster recovery}
All Raw data is also at \gls{SLAC} and can be resent over a period of time.

\subsection{UKDF Processing  \gls{Enclave}}
25\% of processing will be done on \gls{IRIS}.
\gls{ROE} staff are responsible for the \gls{UKDF} noting that \gls{IRIS} is a shared computing facility beyond their control.

\subsubsection{Threats and Security infrastructure}
The considerations for this Enclave are a combination of those for the Offline Production Enclave and \gls{Archive} Enclave.
Access will be granted only to \gls{UKDF} staff and Rubin staff.
UKDF have their own cyber security procedures which will be adhered to.
\subsubsection{Disaster recovery}
All Raw data is at \gls{IN2P3} and can be resent over a period of time.


\subsection{External entities}
There are a number of \gls{IDAC}s which will have and serve catalogs and or images.
These are within our realm of security to some extent but not entirely - we rely on trust at some level.
\subsubsection{Threats and Security infrastructure}
The obvious threat here is unauthorized access to the data rights accessible data.
Any \gls{IDAC} must adhere to our user access protocols so this should not happen.
If unauthorized access occurs the impact is low in terms of system integrity - it may reflect badly on Rubin Observatory and erode the brand and the entire notion of restricted access to the data.

\subsubsection{Disaster recovery}
We are not concerned with disasters at IDACs.
We can resend the appropriate data to them.

\section{Telescope and Site  System architecture} \label{sec:tsarc}

 We concern our selves here mainly with the software architecture of telescope and site,
this includes the control system but also the controlled devices and various test stands.

The control system architecture is given in \citeds{LSE-150}.
Broadly this is a message bus architecture with  various controllable components such as the Camera, Environmental Control, etc. attached to it.
The components can receive control messages and telemetry from the bus by listening to various queues.
The script queue component allows for orchestrated commanding of various components.

This set may be seen for the Main Telescope as well as the Auxiliary Telescope.
In addition there is a test stand in the Base and one in Tucson which have physical DAQ hardware to emulate the camera and can simulate many other physical components for testing the control system.

We consider these systems under the same headings used in \secref{sec:dparc}.





\subsection{Summit Control System}
\subsubsection{Threats and Security infrastructure}
\subsubsection{Disaster recovery}

\subsection{Base Test Stand}
\subsubsection{Threats and Security infrastructure}
\subsubsection{Disaster recovery}

\subsection{Tucson Test Stand}
\subsubsection{Threats and Security infrastructure}
\subsubsection{Disaster recovery}


\section{Rubin Directors Office} \label{sec:rdo}

The directors office is in Tucson Arizona and hosts several observatory functions.
THese include Jira, Conflunece, Docushare  as well as  engineering oriented serives such as


Two plans exist for Tucson:

\begin{itemize}
\item “LSST Tucson Site Disaster Recovery Plan” \citedsp{LPM-101}
\item “LSST Tucson Site IT Cybersecurity Policy" \citedsp{LPM-125}
\end{itemize}

