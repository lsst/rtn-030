% DO NOT EDIT - generated by /Users/womullan/LSSTgit/lsst-texmf/bin/generateAcronyms.py from https://lsst-texmf.lsst.io/.
\newacronym{API} {API} {Application Programming Interface}
\newacronym{AURA} {AURA} {\gls{Association of Universities for Research in Astronomy}}
\newacronym{AVS} {AVS} {Alert Vetting System}
\newglossaryentry{Alert} {name={Alert}, description={A packet of information for each source detected with signal-to-noise ratio > 5 in a difference image by Alert Production, containing measurement and characterization parameters based on the past 12 months of LSST observations plus small cutouts of the single-visit, template, and difference images, distributed via the internet}}
\newglossaryentry{Alert Production} {name={Alert Production}, description={Executing on the Prompt Processing system, the Alert Production payload processes and calibrates incoming images, performs Difference Image Analysis to identify DIASources and DIAObjects, and then packages the resulting alerts for distribution.}}
\newglossaryentry{Archive} {name={Archive}, description={The repository for documents required by the NSF to be kept. These include documents related to design and development, construction, integration, test, and operations of the LSST observatory system. The archive is maintained using the enterprise content management system DocuShare, which is accessible through a link on the project website www.project.lsst.org}}
\newglossaryentry{Archive Center} {name={Archive Center}, description={Part of the LSST Data Management System, the LSST archive center is a data center at NCSA that hosts the LSST Archive, which includes released science data and metadata, observatory and engineering data, and supporting software such as the LSST Software Stack}}
\newglossaryentry{Association of Universities for Research in Astronomy} {name={Association of Universities for Research in Astronomy}, description={ consortium of US institutions and international affiliates that operates world-class astronomical observatories, AURA is the legal entity responsible for managing what it calls independent operating Centers, including LSST, under respective cooperative agreements with the National Science Foundation. AURA assumes fiducial responsibility for the funds provided through those cooperative agreements. AURA also is the legal owner of the AURA Observatory properties in Chile}}
\newglossaryentry{Authentication} {name={Authentication}, description={The action of demonstrating who you are and an person, mission, or other entity. Usually by use of a password or security token}}
\newglossaryentry{Authorization} {name={Authorization}, description={The action of allowing an authorized or anonymous entity access to data or services.}}
\newacronym{BTS} {BTS} {Base (La Serena) Test Stand}
\newglossaryentry{Baseline} {name={Baseline}, description={The point at which project designs or requirements are considered to be 'frozen' and after which all changes must be traced and approved}}
\newglossaryentry{Broker} {name={Broker}, description={Software which receives and redistributes Alerts, and may also perform processing such as filtering for certain characteristics, cross-matching with non-LSST catalogs, and/or light-curve classification, in order to identify and prioritize targets for follow-up and/or make scientific analyses. }}
\newglossaryentry{Butler} {name={Butler}, description={A middleware component for persisting and retrieving image datasets (raw or processed), calibration reference data, and catalogs}}
\newacronym{CCB} {CCB} {\gls{Change Control Board}}
\newacronym{CCD} {CCD} {\gls{Charge-Coupled Device}}
\newacronym{COS} {COS} {Center \gls{Operations} Services}
\newacronym{CS} {CS} {citizen science}
\newacronym{CSC} {CSC} {Commandable \gls{SAL} Component}
\newacronym{CUI} {CUI} {Controlled Unclassified Information}
\newglossaryentry{Camera} {name={Camera}, description={The LSST subsystem responsible for the 3.2-gigapixel LSST camera, which will take more than 800 panoramic images of the sky every night. SLAC leads a consortium of Department of Energy laboratories to design and build the camera sensors, optics, electronics, cryostat, filters and filter exchange mechanism, and camera control system}}
\newglossaryentry{Center} {name={Center}, description={An entity managed by AURA that is responsible for execution of a federally funded project}}
\newglossaryentry{Change Control} {name={Change Control}, description={The systematic approach to managing all changes to the LSST system, including technical data and policy documentation. The purpose is to ensure that no unnecessary changes are made, all changes are documented, and resources are used efficiently and appropriately}}
\newglossaryentry{Change Control Board} {name={Change Control Board}, description={Advisory board to the Project Manager; composed of technical and management representatives who recommend approval or disapproval of proposed changes to, deviations from, and waivers to a configuration item's current approved configuration documentation}}
\newglossaryentry{Charge-Coupled Device} {name={Charge-Coupled Device}, description={a particular kind of solid-state sensor for detecting optical-band photons. It is composed of a 2-D array of pixels, and one or more read-out amplifiers}}
\newacronym{ComCam} {ComCam} {The commissioning \gls{camera} is a single-raft, 9-CCD \gls{camera} that will be installed in LSST during commissioning, before the final \gls{camera} is ready.}
\newglossaryentry{Commissioning} {name={Commissioning}, description={A two-year phase at the end of the Construction project during which a technical team a) integrates the various technical components of the three subsystems; b) shows their compliance with ICDs and system-level requirements as detailed in the LSST Observatory System Specifications document (OSS, LSE-30); and c) performs science verification to show compliance with the survey performance specifications as detailed in the LSST Science Requirements Document (SRD, LPM-17)}}
\newglossaryentry{Compliance} {name={Compliance}, description={Adherence to the laws, regulations, award terms and conditions, specifications, and internal policies applicable to the LSST Project}}
\newglossaryentry{Construction} {name={Construction}, description={The period during which LSST observatory facilities, components, hardware, and software are built, tested, integrated, and commissioned. Construction follows design and development and precedes operations. The LSST construction phase is funded through the NSF MREFC account}}
\newacronym{DAC} {DAC} {\gls{Data Access Center}}
\newacronym{DAQ} {DAQ} {Data Acquisition System}
\newacronym{DCR} {DCR} {\gls{Differential Chromatic Refraction}}
\newacronym{DE} {DE} {dark energy}
\newacronym{DIA} {DIA} {\gls{Difference Image Analysis}}
\newglossaryentry{DIAObject} {name={DIAObject}, description={A DIAObject is the association of DIASources, by coordinate, that have been detected with signal-to-noise ratio greater than 5 in at least one difference image. It is distinguished from a regular Object in that its brightness varies in time, and from a SSObject in that it is stationary (non-moving)}}
\newglossaryentry{DIASource} {name={DIASource}, description={A DIASource is a detection with signal-to-noise ratio greater than 5 in a difference image}}
\newacronym{DM} {DM} {\gls{Data Management}}
\newacronym{DMS} {DMS} {\gls{Data Management Subsystem}}
\newacronym{DMTN} {DMTN} {DM Technical Note}
\newacronym{DOE} {DOE} {\gls{Department of Energy}}
\newacronym{DP} {DP} {Data Production}
\newacronym{DR} {DR} {\gls{Data Release}}
\newacronym{DRP} {DRP} {\gls{Data Release Production}}
\newacronym{DWDM} {DWDM} {Dense Wave Division Multiplex}
\newglossaryentry{Data Access Center} {name={Data Access Center}, description={Part of the LSST Data Management System, the US and Chilean DACs will provide authorized access to the released LSST data products, software such as the Science Platform, and computational resources for data analysis. The US DAC also includes a service for distributing bulk data on daily and annual (Data Release) timescales to partner institutions, collaborations, and LSST Education and Public Outreach (EPO). }}
\newglossaryentry{Data Management} {name={Data Management}, description={The LSST Subsystem responsible for the Data Management System (DMS), which will capture, store, catalog, and serve the LSST dataset to the scientific community and public. The DM team is responsible for the DMS architecture, applications, middleware, infrastructure, algorithms, and Observatory Network Design. DM is a distributed team working at LSST and partner institutions, with the DM Subsystem Manager located at LSST headquarters in Tucson}}
\newglossaryentry{Data Management Subsystem} {name={Data Management Subsystem}, description={The Data Management Subsystem is one of the four subsystems which constitute the LSST Construction Project. The Data Management Subsystem is responsible for developing and delivering the LSST Data Management System to the LSST Operations Project}}
\newglossaryentry{Data Management System} {name={Data Management System}, description={The computing infrastructure, middleware, and applications that process, store, and enable information extraction from the LSST dataset; the DMS will process peta-scale data volume, convert raw images into a faithful representation of the universe, and archive the results in a useful form. The infrastructure layer consists of the computing, storage, networking hardware, and system software. The middleware layer handles distributed processing, data access, user interface, and system operations services. The applications layer includes the data pipelines and the science data archives' products and services}}
\newglossaryentry{Data Release} {name={Data Release}, description={The approximately annual reprocessing of all LSST data, and the installation of the resulting data products in the LSST Data Access Centers, which marks the start of the two-year proprietary period}}
\newglossaryentry{Data Release Processing} {name={Data Release Processing}, description={Deprecated term; see Data Release Production}}
\newglossaryentry{Data Release Production} {name={Data Release Production}, description={An episode of (re)processing all of the accumulated LSST images, during which all output DR data products are generated. These episodes are planned to occur annually during the LSST survey, and the processing will be executed at the Archive Center. This includes Difference Imaging Analysis, generating deep Coadd Images, Source detection and association, creating Object and Solar System Object catalogs, and related metadata}}
\newglossaryentry{Department of Energy} {name={Department of Energy}, description={cabinet department of the United States federal government; the DOE has assumed technical and financial responsibility for providing the LSST camera. The DOE's responsibilities are executed by a collaboration led by SLAC National Accelerator Laboratory}}
\newglossaryentry{Difference Image} {name={Difference Image}, description={Refers to the result formed from the pixel-by-pixel difference of two images of the sky, after warping to the same pixel grid, scaling to the same photometric response, matching to the same PSF shape, and applying a correction for Differential Chromatic Refraction. The pixels in a difference thus formed should be zero (apart from noise) except for sources that are new, or have changed in brightness or position. In the LSST context, the difference is generally taken between a visit image and template. }}
\newglossaryentry{Difference Image Analysis} {name={Difference Image Analysis}, description={The detection and characterization of sources in the Difference Image that are above a configurable threshold, done as part of Alert Generation Pipeline}}
\newglossaryentry{Differential Chromatic Refraction} {name={Differential Chromatic Refraction}, description={The refraction of incident light by Earth's atmosphere causes the apparent position of objects to be shifted, and the size of this shift depends on both the wavelength of the source and its airmass at the time of observation. DCR corrections are done as a part of DIA}}
\newglossaryentry{Director} {name={Director}, description={The person responsible for the overall conduct of the project; the LSST director is charged with ensuring that both the scientific goals and management constraints on the project are met. S/he is the principal public spokesperson for the project in all matters and represents the project to the scientific community, AURA, the member institutions of LSSTC, and the funding agencies}}
\newglossaryentry{DocuShare} {name={DocuShare}, description={The trade name for the enterprise management software used by LSST to archive and manage documents}}
\newglossaryentry{Document} {name={Document}, description={Any object (in any application supported by DocuShare or design archives such as PDMWorks or GIT) that supports project management or records milestones and deliverables of the LSST Project}}
\newacronym{Duo} {Duo} {2 factor authentication system}
\newacronym{EPO} {EPO} {\gls{Education and Public Outreach}}
\newglossaryentry{Education and Public Outreach} {name={Education and Public Outreach}, description={The LSST subsystem responsible for the cyberinfrastructure, user interfaces, and outreach programs necessary to connect educators, planetaria, citizen scientists, amateur astronomers, and the general public to the transformative LSST dataset}}
\newglossaryentry{Enclave} {name={Enclave}, description={Individually defined portions of the computational resources at the Summit, Base, NCSA, and Satellite Facilities, such as the Prompt Enclave, the Archive Enclave, etc. }}
\newacronym{FIPS} {FIPS} {Federal Information Processing Standards}
\newacronym{FITS} {FITS} {\gls{Flexible Image Transport System}}
\newacronym{FRDF} {FRDF} {French Data Facility}
\newglossaryentry{Flexible Image Transport System} {name={Flexible Image Transport System}, description={an international standard in astronomy for storing images, tables, and metadata in disk files. See the IAU FITS Standard for details}}
\newacronym{FrDF} {FrDF} {French Data Facility}
\newacronym{FreeIPA} {FreeIPA} {is an integrated security information management solution}
\newacronym{HEP} {HEP} { High Energy Physics}
\newacronym{HTTP} {HTTP} {HyperText Transfer Protocol}
\newacronym{HVAC} {HVAC} {Heating, Ventilation, and Air Conditioning}
\newglossaryentry{Handle} {name={Handle}, description={The unique identifier assigned to a document uploaded to DocuShare}}
\newacronym{IAU} {IAU} {International Astronomical Union}
\newacronym{IDAC} {IDAC} {\gls{Independent Data Access Center}}
\newacronym{IN2P3} {IN2P3} {Institut National de Physique Nucléaire et de Physique des Particules}
\newacronym{IPA} {IPA} {FreeIPA - Identity, Policy, Audit}
\newacronym{IRIS} {IRIS} {e-Infrastructure for Research and Innovation for \gls{STFC}}
\newacronym{ISO} {ISO} {Informaiton Security Officer}
\newacronym{IT} {IT} {Information Technology}
\newacronym{IaC} {IaC} {Infrastructure as Code}
\newglossaryentry{Incident} {name={Incident}, description={An undesired event, which under slightly different circumstances, could have resulted in harm to people, damage to property, or loss to process}}
\newglossaryentry{Independent Data Access Center} {name={Independent Data Access Center}, description={Externally supported and administered versions of the DAC to serve the full, or a limited subset of, the LSST data products and/or software to authorized users. }}
\newacronym{JPL} {JPL} {Jet Propulsion Laboratory (\gls{DE} ephemerides)}
\newacronym{LDM} {LDM} {LSST Data Management (Document \gls{Handle})}
\newacronym{LDO} {LDO} {LSST Document \gls{Operations} (Document Handle)}
\newacronym{LHN} {LHN} {long haul network}
\newacronym{LLNL} {LLNL} {Lawrence Livermore National Laboratory}
\newacronym{LPM} {LPM} {LSST Project Management (Document \gls{Handle})}
\newacronym{LSE} {LSE} {LSST \gls{Systems Engineering} (Document Handle)}
\newacronym{LSST} {LSST} {Legacy Survey of Space and Time (formerly Large Synoptic Survey Telescope)}
\newglossaryentry{LSST Camera} {name={LSST Camera}, description={3.2 Gigapixel camera and lens system build by SLAC to perform the Legacy Survey of Space and Time.}}
\newglossaryentry{LSST Corporation} {name={LSST Corporation}, description={An Arizona 501(c)3 not-for-profit corporation formed in 2003 for the purpose of designing, constructing, and operating the LSST System. During design and development, the Corporation stewarded private funding used for such essential contributions as early site preparation, mirror construction, and early data management system development. During construction, LSSTC will secure private operations funding from international affiliates and play a key role in preparing the scientific community to use the LSST dataset}}
\newglossaryentry{LSST Project Office} {name={LSST Project Office}, description={Official name of the stand-alone AURA operating center responsible for execution of the LSST construction project under the NSF MREFC account}}
\newglossaryentry{LSST Science Pipelines} {name={LSST Science Pipelines}, description={software used to perform the LSST data reduction pipelines.lsst.io}}
\newacronym{LSSTC} {LSSTC} {\gls{LSST Corporation}}
\newacronym{LSSTPO} {LSSTPO} {\gls{LSST Project Office}}
\newacronym{MPC} {MPC} {Minor Planet \gls{Center}}
\newacronym{MREFC} {MREFC} {\gls{Major Research Equipment and Facility Construction}}
\newglossaryentry{Major Research Equipment and Facility Construction} {name={Major Research Equipment and Facility Construction}, description={the NSF account through which large facilities construction projects such as LSST are funded}}
\newacronym{NAT} {NAT} {Network Address Translation}
\newacronym{NCSA} {NCSA} {National \gls{Center} for Supercomputing Applications}
\newacronym{NEO} {NEO} {Near-Earth \gls{Object}}
\newacronym{NIST} {NIST} {National Institute of Standards and Technology (USA)}
\newacronym{NOAO} {NOAO} {National Optical Astronomy Observatories now \gls{NOIRLab}}
\newacronym{NOIRLab} {NOIRLab} {NSF's National Optical-Infrared Astronomy Research Laboratory; \url{https://noirlab.edu}}
\newacronym{NSF} {NSF} {\gls{National Science Foundation}}
\newacronym{NTS} {NTS} {NCSA Test Stand}
\newglossaryentry{National Science Foundation} {name={National Science Foundation}, description={primary federal agency supporting research in all fields of fundamental science and engineering; NSF selects and funds projects through competitive, merit-based review}}
\newacronym{OGA} {OGA} {Other Government Agencies}
\newacronym{OPS} {OPS} {\gls{Operations}}
\newacronym{OS} {OS} {Operating System}
\newacronym{OSS} {OSS} {Observatory System Specifications; \gls{LSE}-30}
\newacronym{OUO} {OUO} {Official Use Only}
\newglossaryentry{Object} {name={Object}, description={In LSST nomenclature this refers to an astronomical object, such as a star, galaxy, or other physical entity. E.g., comets, asteroids are also Objects but typically called a Moving Object or a Solar System Object (SSObject). One of the DRP data products is a table of Objects detected by LSST which can be static, or change brightness or position with time}}
\newglossaryentry{Operations} {name={Operations}, description={The 10-year period following construction and commissioning during which the LSST Observatory conducts its survey}}
\newacronym{PMO} {PMO} {\gls{Project Management Office}}
\newacronym{PSF} {PSF} {Point Spread Function}
\newglossaryentry{Primavera} {name={Primavera}, description={The trade name for the project management software suite used by LSST to maintain its program plan and schedule}}
\newglossaryentry{Project Management Office} {name={Project Management Office}, description={the work element responsible for achieving the project's objectives}}
\newglossaryentry{Project Manager} {name={Project Manager}, description={The person responsible for exercising leadership and oversight over the entire Rubin project; he or she controls schedule, budget, and all contingency funds}}
\newglossaryentry{Prompt Processing} {name={Prompt Processing}, description={The data processing which occurs at the Archive Center based on the stream of images coming from the telescope. This includes both Alert Production, which scans the image stream to identify and send alerts on transient and variable sources, and Solar System Processing, which identifies and characterizes objects in our solar system. It also includes specialized rapid calibration and Commissioning processing. Prompt Processing generates the Prompt Data Products.}}
\newacronym{QA} {QA} {\gls{Quality Assurance}}
\newacronym{QC} {QC} {\gls{Quality Control}}
\newglossaryentry{Quality Assurance} {name={Quality Assurance}, description={All activities, deliverables, services, documents, procedures or artifacts which are designed to ensure the quality of DM deliverables. This may include QC systems, in so far as they are covered in the charge described in LDM-622. Note that contrasts with the LDM-522 definition of “QA” as “Quality Analysis”, a manual process which occurs only during commissioning and operations. See also: Quality Control}}
\newglossaryentry{Quality Control} {name={Quality Control}, description={Services and processes which are aimed at measuring and monitoring a system to verify and characterize its performance (as in LDM-522). Quality Control systems run autonomously, only notifying people when an anomaly has been detected. See also Quality Assurance}}
\newacronym{RDM} {RDM} {Rubin \gls{Data Management}}
\newacronym{RDO} {RDO} {Rubin Directors Office}
\newacronym{RDP} {RDP} {Rubin Data Production (Obsolete use \gls{RDM})}
\newacronym{ROE} {ROE} {Royal Observatory Edinburgh}
\newacronym{RPF} {RPF} {Rubin system PerFormance}
\newacronym{RSP} {RSP} {Rubin \gls{Science Platform}}
\newacronym{RTN} {RTN} {Rubin Technical Note}
\newglossaryentry{Review} {name={Review}, description={Programmatic and/or technical audits of a given component of the project, where a preferably independent committee advises further project decisions, based on the current status and their evaluation of it. The reviews assess technical performance and maturity, as well as the compliance of the design and end product with the stated requirements and interfaces}}
\newglossaryentry{Risk} {name={Risk}, description={The degree of exposure to an event that might happen to the detriment of a program, project, or other activity. It is described by a combination of the probability that the risk event will occur and the consequence of the extent of loss from the occurrence, or impact. Risk is an inherent part of all activities, whether the activity is simple and small, or large and complex}}
\newglossaryentry{Risk Management} {name={Risk Management}, description={The art and science of planning, assessing, and handling future events to avoid unfavorable impacts on project cost, schedule, or performance to the extent possible. Risk management is a structured, formal, and disciplined activity focused on the necessary steps and planning actions to determine and control risks to an acceptable level. Risk Management is an event-based management approach to managing uncertainty}}
\newacronym{SAL} {SAL} {Service Abstraction Layer}
\newacronym{SLAC} {SLAC} {\gls{SLAC National Accelerator Laboratory}}
\newglossaryentry{SLAC National Accelerator Laboratory} {name={SLAC National Accelerator Laboratory}, description={A national laboratory funded by the US Department of Energy (DOE); SLAC leads a consortium of DOE laboratories that has assumed responsibility for providing the LSST camera. Although the Camera project manages its own schedule and budget, including contingency, the Camera team’s schedule and requirements are integrated with the larger Project.  The camera effort is accountable to the LSSTPO.}}
\newacronym{SP} {SP} {Survey Performance}
\newacronym{SQR} {SQR} {SQuARE document handle}
\newacronym{SRD} {SRD} {LSST Science Requirements; \gls{LPM}-17}
\newacronym{SSID} {SSID} {Service Set Identifier}
\newacronym{SSL} {SSL} {Secure Sockets Layer}
\newacronym{STFC} {STFC} {UK Science and Technology Facilities Council}
\newglossaryentry{Science Pipelines} {name={Science Pipelines}, description={The library of software components and the algorithms and processing pipelines assembled from them that are being developed by DM to generate science-ready data products from LSST images. The Pipelines may be executed at scale as part of LSST Prompt or Data Release processing, or pieces of them may be used in a standalone mode or executed through the Rubin Science Platform. The Science Pipelines are one component of the LSST Software Stack}}
\newglossaryentry{Science Platform} {name={Science Platform}, description={A set of integrated web applications and services deployed at the LSST Data Access Centers (DACs) through which the scientific community will access, visualize, and perform next-to-the-data analysis of the LSST data products}}
\newglossaryentry{Software Stack} {name={Software Stack}, description={Often referred to as the LSST Stack, or just The Stack, it is the collection of software written by the LSST Data Management Team to process, generate, and serve LSST images, transient alerts, and catalogs. The Stack includes the LSST Science Pipelines, as well as packages upon which the DM software depends. It is open source and publicly available}}
\newglossaryentry{Solar System Object} {name={Solar System Object}, description={A solar system object is an astrophysical object that is identified as part of the Solar System: planets and their satellites, asteroids, comets, etc. This class of object had historically been referred to within the LSST Project as Moving Objects}}
\newglossaryentry{Solar System Processing} {name={Solar System Processing}, description={A component of the Prompt Processing system, Solar System Processing identifies new SSObjects using unassociated DIASources.}}
\newglossaryentry{Source} {name={Source}, description={A single detection of an astrophysical object in an image, the characteristics for which are stored in the Source Catalog of the DRP database. The association of Sources that are non-moving lead to Objects; the association of moving Sources leads to Solar System Objects. (Note that in non-LSST usage "source" is often used for what LSST calls an Object.)}}
\newglossaryentry{Subsystem} {name={Subsystem}, description={A set of elements comprising a system within the larger LSST system that is responsible for a key technical deliverable of the project}}
\newglossaryentry{Subsystem Manager} {name={Subsystem Manager}, description={responsible manager for an LSST subsystem; he or she exercises authority, within prescribed limits and under scrutiny of the Project Manager, over the relevant subsystem's cost, schedule, and work plans}}
\newglossaryentry{Summit} {name={Summit}, description={The site on the Cerro Pach\'{o}n, Chile mountaintop where the LSST observatory, support facilities, and infrastructure will be built}}
\newglossaryentry{Systems Engineering} {name={Systems Engineering}, description={an interdisciplinary field of engineering that focuses on how to design and manage complex engineering systems over their life cycles. Issues such as requirements engineering, reliability, logistics, coordination of different teams, testing and evaluation, maintainability and many other disciplines necessary for successful system development, design, implementation, and ultimate decommission become more difficult when dealing with large or complex projects. Systems engineering deals with work-processes, optimization methods, and risk management tools in such projects. It overlaps technical and human-centered disciplines such as industrial engineering, control engineering, software engineering, organizational studies, and project management. Systems engineering ensures that all likely aspects of a project or system are considered, and integrated into a whole}}
\newacronym{TLS} {TLS} {Transport Layer Security}
\newacronym{TMA} {TMA} {Telescope Mount Assembly}
\newacronym{TTS} {TTS} {Tucson Test Stand}
\newglossaryentry{Telescope and Site} {name={Telescope and Site}, description={The LSST subsystem responsible for design and construction of the telescope structure, telescope mirrors, optical wavefront measurement and control system, telescope and observatory control systems software, and the summit and base facilities.}}
\newacronym{UK} {UK} {United Kingdom}
\newacronym{UKDF} {UKDF} {United Kingdom Data Facility}
\newacronym{US} {US} {United States}
\newacronym{USB} {USB} {Universal Serial Bus}
\newacronym{USDF} {USDF} {United States Data Facility}
\newacronym{UTC} {UTC} {Coordinated Universal Time}
\newacronym{VLAN} {VLAN} { Virtual Local Area Network}
\newacronym{VPN} {VPN} {virtual private network}
\newacronym{VRO} {VRO} {(not to be used)Vera C. Rubin Observatory}
\newacronym{WBS} {WBS} {\gls{Work Breakdown Structure}}
\newglossaryentry{Work Breakdown Structure} {name={Work Breakdown Structure}, description={a tool that defines and organizes the LSST project's total work scope through the enumeration and grouping of the project's discrete work elements}}
\newglossaryentry{airmass} {name={airmass}, description={The pathlength of light from an astrophysical source through the Earth's atmosphere. It is given approximately by sec z, where z is the angular distance from the zenith (the point directly overhead, where airmass = 1.0) to the source}}
\newglossaryentry{astronomical object} {name={astronomical object}, description={A star, galaxy, asteroid, or other physical object of astronomical interest. Beware: in non-LSST usage, these are often known as sources}}
\newglossaryentry{background} {name={background}, description={In an image, the background consists of contributions from the sky (e.g., clouds or scattered moonlight), and from the telescope and camera optics, which must be distinguished from the astrophysical background. The sky and instrumental backgrounds are characterized and removed by the LSST processing software using a low-order spatial function whose coefficients are recorded in the image metadata}}
\newglossaryentry{calibration} {name={calibration}, description={The process of translating signals produced by a measuring instrument such as a telescope and camera into physical units such as flux, which are used for scientific analysis. Calibration removes most of the contributions to the signal from environmental and instrumental factors, such that only the astronomical component remains}}
\newglossaryentry{camera} {name={camera}, description={An imaging device mounted at a telescope focal plane, composed of optics, a shutter, a set of filters, and one or more sensors arranged in a focal plane array}}
\newglossaryentry{cloud} {name={cloud}, description={A visible mass of condensed water vapor floating in the atmosphere, typically high above the ground or in interstellar space acting as the birthplace for stars.  Also a way of computing (on other peoples computers leveraging their services and availability).}}
\newglossaryentry{configuration} {name={configuration}, description={A task-specific set of configuration parameters, also called a 'config'. The config is read-only; once a task is constructed, the same configuration will be used to process all data. This makes the data processing more predictable: it does not depend on the order in which items of data are processed. This is distinct from arguments or options, which are allowed to vary from one task invocation to the next}}
\newglossaryentry{element} {name={element}, description={A node in the hierarchical project WBS}}
\newglossaryentry{flux} {name={flux}, description={Shorthand for radiative flux, it is a measure of the transport of radiant energy per unit area per unit time. In astronomy this is usually expressed in cgs units: erg/cm2/s}}
\newglossaryentry{metadata} {name={metadata}, description={General term for data about data, e.g., attributes of astronomical objects (e.g. images, sources, astroObjects, etc.) that are characteristics of the objects themselves, and facilitate the organization, preservation, and query of data sets. (E.g., a FITS header contains metadata)}}
\newglossaryentry{middleware} {name={middleware}, description={Software that acts as a bridge between other systems or software usually a database or network. Specifically in the Data Management System this refers to Butler for data access and Workflow management for distributed processing.}}
\newglossaryentry{monitoring} {name={monitoring}, description={In DM QA, this refers to the process of collecting, storing, aggregating and visualizing metrics}}
\newglossaryentry{nublado} {name={nublado}, description={The service underpinning the Notebook Aspect of the Rubin Science Platform}}
\newglossaryentry{patch} {name={patch}, description={An quadrilateral sub-region of a sky tract, with a size in pixels chosen to fit easily into memory on desktop computers}}
\newglossaryentry{script queue} {name={script queue}, description={A CSC which manages SAL scripts, running one script at a time until the queue is exhausted or paused}}
\newglossaryentry{shape} {name={shape}, description={In reference to a Source or Object, the shape is a functional characterization of its spatial intensity distribution, and the integral of the shape is the flux. Shape characterizations are a data product in the DIASource, DIAObject, Source, and Object catalogs}}
\newglossaryentry{sky map} {name={sky map}, description={A sky tessellation for LSST. The Stack includes software to define a geometric mapping from the representation of World Coordinates in input images to the LSST sky map. This tessellation is comprised of individual tracts which are, in turn, comprised of patches}}
\newglossaryentry{software} {name={software}, description={The programs and other operating information used by a computer.}}
\newglossaryentry{test stand} {name={test stand}, description={An environment used for testing the operation of the LSST Camera, or some component thereof. In the Data Management context, this generally refers to a simulated Camera readout system used to test the interface between the Camera and the DM system (see, for example, NTS)}}
\newglossaryentry{tract} {name={tract}, description={A portion of sky, a spherical convex polygon, within the LSST all-sky tessellation (sky map). Each tract is subdivided into sky patches}}
\newglossaryentry{transient} {name={transient}, description={A transient source is one that has been detected on a difference image, but has not been associated with either an astronomical object or a solar system body}}
