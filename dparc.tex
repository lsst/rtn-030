%ENclave text from KT  https://docs.google.com/document/d/1apNSWtIpbS7aCitaF_K0JXLAAY9PvCEmmvnNvHmecdk/edit#heading=h.nwgy0freb3zv


\section{Data Management system architecture} \label{sec:dparc}
The overall system architecture is available in \citeds{LDM-148}.
Details on the \gls{USDF} specifications are given in \citeds{DMTN-189}.

\secref{sec:desc} gives a high level overview of the system, architecturally we look
at this as a set of enclaves.
As images are processed in the Prompt and Offline Production enclaves, their resulting data products are stored in the \gls{Archive} enclave and made available to the DAC enclave where data rights holders can access and analyze them.
In addition, Rubin Observatory staff will use the Development/Integration enclave to maintain the Observatory's software tools and systems and to develop new versions of them.

These enclaves are further described here and for each a series of subsections explore :


\begin{enumerate}
\item Threats and Security infrastructure
\item Disaster recovery
\end{enumerate}

\subsection{US Data Facility}\label{sec:usdf}

A number of enclaves are within the the USDF.
Maintenance and Security of these system falls to SLAC IT.

\subsubsection{Prompt \gls{Enclave}} \label{sec:promptenc}

The Prompt enclave receives images from the Observatory facilities in Chile via a Long Haul Network connection.
It stores these and processes them into Prompt data products of three main types:
alerts for things that have moved or changed, measurement catalogs, and processed images.
The alerts can be further subdivided into \emph{streak} alerts for objects that have moved a long distance and \emph{non-streak} alerts for all other objects.
Measurements in the catalogs follow the same subdivision. Images may be \gls{Commissioning} images used for testing and characterizing the Observatory systems, normal science images without significant \emph{streaks}, or delayed science images that do contain significant \emph{streaks}.

\emph{Streak} alerts found in the \gls{OUO} catalog at SLAC will not be released.
Uncatalogued \emph{Streak} alerts agreed with \gls{JPL} \gls{NEO} group  are to be published at least to \gls{MPC}.
\emph{Non-streak} alerts are to be published to the world at large within 60 seconds of the original raw image being taken.
Normal science images are made available to data rights holders in the \gls{DAC} after an 80 hour embargo period.
Delayed science images, as identified by the agencies, are released after their specific embargo period. Commissioning images are made available to data rights holders after a 30 day embargo period.

All Prompt data products are checked for quality by automated systems but also by human operators from the Rubin Observatory staff, who have access to all images and data products in order to perform spot checks or follow ups.

\paragraph{ Threats and Security infrastructure}
The obvious threat surfaces here are :
\begin{enumerate}
\item Transmission of Data from Chile. IPSec built into the routers will be used on the \gls{LHN}. \citeds{DMTN-108} discusses threats in this realm a little more.
\item Transmission to \gls{LLNL}.  This will be over internet using TLS.
\item Staff access for \gls{QA}. All the usual user threats such as phishing apply - these users are however governed bu SLAC security policies\footnote{\url{https://it.slac.stanford.edu/cybersecurity/compliance}}.
\item QA tools. The web accessible QA tools should have a threat analysis performed by SLAC or our Security consultants although they will be behind SLAC \gls{VPN} and 2FA. These tools are also only accessible by staff and probably pose a low risk.
\item The for the initial 10 days (30 in commissioning) after acquisition the data is maintain on encrypted disks with in the restricted \gls{OGA} rack.
\end{enumerate}

\paragraph{Disaster recovery}
All embargoed data is also stored on a secure server in Chile hence it can be retransmitted as needed.
In the case of a total wipe out of the \gls{OGA} systems use of Chef, docker etc allow redeployment rapidly.
See also the \gls{SLAC} Rubin disaster recovery plan \citeds{RTN-078}.
The back up for the embargoed data is the Chile OGA rack.

\subsubsection{Satellite Catalog Sub-\gls{Enclave}} \label{sec:satcat}
This enclave holds the only \gls{OUO} data on the project.
\paragraph{ Threats and Security infrastructure}
\paragraph{Disaster recovery}

\subsubsection{ Offline Production  \gls{Enclave}}
Each year (or more frequently), the Offline Production enclave takes the raw images accumulated to date in the Archive and reprocesses them to generate highly accurate, consistent images and measurement catalogs, known as a Data Release. These data products are stored in the Archive and made available to data rights holders in the DAC after they have been checked by automated systems and after Rubin Observatory staff has vetted, characterized, and documented them. Offline Production is split between the USDF and the \gls{FrDF} and UKDF. Each Data Facility performs part of the computations and exchanges its results with the others, so all have a complete set of data products at release time.


\paragraph{ Threats and Security infrastructure}
\begin{itemize}
\item Offline production data is no longer embargoed ergo not considered under threat.
\item Although data exchange among the facilities uses encryption (secure HTTP over TLS), if data were to be intercepted in transfer between sites, this could only occur after the embargo period hence the security risk is low.
\item Malicious users could disrupt data or processing.
We are using standard tooling from \gls{HEP} which has been in use for many years and gives a level confidence of their suitability in this scientific endeavor.
Still internal users remain a major risk - we maintain an inclusive project and try to avoid disgruntled team members.
\end{itemize}
\paragraph{Disaster recovery}
Post embargo FrDF keeps a full copy of the raw data.

\subsubsection{ \gls{Archive}  Enclave}
The raw images, data products, and other records of the survey such as commands, events, and telemetry from Observatory systems are all stored in the \gls{Archive}.
As the permanent scientific record of the survey, no more than 1\% of the raw images or telemetry may be lost or corrupted according to Rubin requirements.

To help ensure this, the French Data Facility maintains a disaster recovery copy of all raw images and selected data products. Additional copies of some raw images and data products will be stored in Observatory systems in Chile.

We intend to have a hybrid access model for the archive where  the \gls{RSP} users and user processing will be on Google cloud while the data resides at SLAC. A cache of images will be held on Google with a dedicated client pulling needed images from SLAC to google.


\paragraph{ Threats and Security infrastructure}

\begin{itemize}
\item Archive data is no longer embargoed ergo not considered under threat.
\item We will have a lot of users which could be problematic. Keeping the data rights holders on the cloud allows a clean separation of concerns between SLAC for processing and archive and the more public facing \gls{RSP}.
\end{itemize}
\paragraph{Disaster recovery}
Post embargo FrDF keeps a full copy of the raw data.



\subsection{ USDF \gls{DAC} Enclave}
The USDF DAC is hosted on Google and  is the responsibility of SLAC.
All deployments on the USDF DAC are made by SQuaRE using Phalanx.

Data rights holders will use the services and systems in this enclave to work with the survey data products.
It is therefore a general-purpose scientific computing facility. Generally users will interact with the Rubin Science Platform (\gls{RSP}), which is composed of a web-based Portal Aspect providing a guided user interface for accessing and analyzing the data, a Notebook Aspect providing an interactive, flexible, programming-oriented interface, and an API Aspect providing an programmable access service.
Users of the DAC may connect from anywhere in the world over the Internet; all such users will be authenticated before accessing any \gls{RSP} service.
The \gls{RSP} is hosted on a cloud service, currently  Google Cloud Platform.

The DAC retrieves the released data products from the \gls{Archive} Enclave via protocols and services authenticated at a service account level only. While end-user identities may be included for audit and accounting purposes, fundamentally the DAC exists to provide access to all \gls{Archive} contents.

\subsubsection{ Threats and Security infrastructure}
The \gls{RSP} is  an attractive generic target due to its computing resources.
There is some user generated data which is mildly sensitive.
Hosting it on a cloud provider reduces risk considerably for the Archive enclave, and also leverages the security products and services made available by the hosting provider.
\citeds{SQR-041} provides a risk assessment for the \gls{RSP}.
\citeds{DMTN-193} provides a more in depth web risk analysis.

Backend archive services could provide another attack surface.
These are governed by \gls{SLAC} security.

\subsubsection{Disaster recovery}
For user  spaces we rely on cloud provider redundancy/backup/recovery.

Our data is cached a full copy is always held at the \gls{USDF} hence any Rubin data at the DAC is expendable.

Further considerations are covered in \citeds{rtn-059}.

\subsection{ Chile \gls{DAC} Enclave}
This proposed DAC in Chile is covered in \cite{LDM-572}.
We will not start work on this until near the start of operations.
\subsubsection{ Threats and Security infrastructure}

\begin{itemize}
\item The Chile DAC is within the Recinto data center and covered by AURA/COS security measures.\item All Rubin traffic is run through a security appliance (currently Zeek).
\item Selected Chilean users have access to the DAC. We will keep the DAC and the users confined with least privileges. We will use a caching mechanism analogous to the Cloud DAC system to restrict access to the object store for the external users.
\item All access will be via \gls{RSP} pods and hence containerized - escalation potential from in side the container will be carefully monitored.
\end{itemize}
\subsubsection{Disaster recovery}
The Chile disaster recovery plan will cover the Chile \gls{DAC} \citeds{ittn-055}.

\subsection{Development and Integration  \gls{Enclave}}
Rubin Observatory and \gls{USDF} staff will use this enclave to build and test new versions of software and services to be deployed in the other enclaves.

\subsubsection{Threats and Security infrastructure}

\begin{itemize}
\item Developers have a higher level of access than most users.
This is a necessary and accepted risk.
\item All developers must adhere to  SLAC access rules e.g. FACTS checking etc.
\end{itemize}

\subsubsection{Disaster recovery}
SLAC keep tape backups.

All code is deployed using Chef or containers and hence fairly easily recoverable in case of catastrophic failure.



\subsection{FrDF Processing  \gls{Enclave}}
40\% of \gls{DRP} will be done at IN2P3.
A full back up of the raw data will also be held there.
The IN2P3 computing infrastructure is described in \url{https://doc.lsst.eu/}.
\subsubsection{Threats and Security infrastructure}
IN2P3 have their own cyber security procedures which will be adhered to.
\subsubsection{Disaster recovery}
All Raw data is also at SLAC and can be resent over a period of time.

\subsection{UKDF Processing  \gls{Enclave}}
25\% of processing will be done on \gls{IRIS}.
\subsubsection{Threats and Security infrastructure}
UKDF have their own cyber security procedures which will be adhered to.
\subsubsection{Disaster recovery}
All Raw data is at IN2P3 and can be resent over a period of time.


\subsection{External entities}
There are a number of \gls{IDAC}s which will have and serve catalogs and or images.
These are within our realm of security to some extend but not entirely - we rely on trust at some level.
\subsubsection{Threats and Security infrastructure}
The obvious threat here is unauthorized access to the data rights accessible data.
Any IDAC must adhere to our user access protocols so this should not happen.
If unauthorized access occurs the impact is low in terms of system integrity - it may reflect badly on Rubin Observatory and erode the brand and the entire notion of restricted access to the data.

\subsubsection{Disaster recovery}
We are not concerned with disasters at IDACs.
We can resend the appropriate data to them.
