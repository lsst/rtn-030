\tiny \begin{longtable} {|p{0.5\textwidth}|p{0.05\textwidth}|p{0.05\textwidth}|p{0.4\textwidth} |} \caption{This table provides an overview of the \citeds{NIST.SP.800-171} and Rubin compliance with it. \label{tab:compliance}}\\ 
\hline 
\textbf{NIST 800-171}&\textbf{2021 Status}&\textbf{Intended Compliance}&\textbf{Note} \\ \hline
{3.1 ACCESS CONTROL}&&& \\ \hline
{3.1.1 Limit system access to authorized users, processes acting on behalf of authorized users, and devices (including other systems).}&{Y}&{Y}& \\ \hline
{3.1.2 Limit system access to the types of transactions and functions that authorized users are permitted to execute.}&{N}&{Y}&{There are many non-administrative users with unrestricted sudo access, this will be addressed.} \\ \hline
{3.1.3 Control the flow of CUI in accordance with approved authorizations.}&{Y}&{Y}& \\ \hline
{3.1.4 Separate the duties of individuals to reduce the risk of malevolent activity without collusion.}&{N}&{Y}&{Principle of least privilege is applied. Many users have access to hosts that is unneeded.} \\ \hline
{3.1.5 Employ the principle of least privilege, including for specific security functions and privileged accounts.}&{N}&{Y}&{Targeted sudo rules are needed for common operations. IPA controls sudo centrally } \\ \hline
{3.1.6 Use non-privileged accounts or roles when accessing nonsecurity functions.}&{Y}&{Y}& \\ \hline
{3.1.7 Prevent non-privileged users from executing privileged functions and capture the execution of such functions in audit logs.}&{}&{Y}&{This is probably sudo attempts audits. Full commands can be logged in at the cost of extra load for the servers.} \\ \hline
{3.1.8 Limit unsuccessful login attempts.}&{N}&{Y}&{I don't believe we do this now but we can; this is not done for ssh on hosts or network equipment.  Web Services such as love, foreman, ipa console, nublado, etc. may need rate limiting [Cristian: we dont use passwords in ssh hosts, it's only ssh keys so technically we are limiting the access to a single attempt. ]} \\ \hline
{3.1.9 Provide privacy and security notices consistent with applicable CUI rules.}&{N}&{Y}&{Check login notices etc. A login banner can be displayed upon login } \\ \hline
{3.1.10 Use session lock with pattern-hiding displays to prevent access and viewing of data after a period of inactivity.}&{Y}&{Y}&{This is our policy.} \\ \hline
{3.1.11 Terminate (automatically) a user session after a defined condition.}&{N}&{Y}&{ssh sessions are generally not limited on hosts; some network equipment has timeouts set; nublado has a session limit for notebooks?} \\ \hline
{3.1.12 Monitor and control remote access sessions.}&{N}&{Y}&{We currently check who and from where is connecting.} \\ \hline
{3.1.13 Employ cryptographic mechanisms to protect the confidentiality of remote access sessions.}&{Y}&{Y}&{VPN is in use} \\ \hline
{3.1.14 Route remote access via managed access control points.}&{N}&{Y}&{Bastion nodes -- LHN is an open back door with no ACLs} \\ \hline
{3.1.15 Authorize remote execution of privileged commands and remote access to security-relevant information.}&{Y}&{Y}& \\ \hline
{3.1.16 Authorize wireless access prior to allowing such connections.}&{Y}&{Y}&{All devics attaching in Chile need to be registered by Mac address.} \\ \hline
{3.1.17 Protect wireless access using authentication and encryption.}&{Y}&{Y}& \\ \hline
{3.1.18 Control connection of mobile devices.}&{Y}&{Y}&{In the sense there is no open wifi, and on the summit devices must be registered. } \\ \hline
{3.1.19 Encrypt CUI on mobile devices and mobile computing platforms.23}&{Y}&{Y}&{Data will not exist on mobile devices - in the case where an image may exist on say commissioning team laptop we will have disk encryption enabled. } \\ \hline
{3.1.20 Verify and control/limit connections to and use of external systems.}&{Y}&{Y}&{This implies vetting of devices that connect to the control network - we use mac address for laptops and personal mobile phones can not connect to the control network. [Cristian: we already have a separation with the LHN SSID and VLANS]} \\ \hline
{3.1.21 Limit use of portable storage devices on external systems.}&{N}&{Y}&{Can be rolled out with puppet but there are some servers that need usb. } \\ \hline
{3.1.22 Control CUI posted or processed on publicly accessible systems.}&{Y}&{Y}&{We do not intend to post images on publicly accessible systems. } \\ \hline
{3.2 AWARENESS AND TRAINING}&&& \\ \hline
{3.2.1 Ensure that managers, systems administrators, and users of organizational systems are made aware of the security risks associated with their activities and of the applicable policies, standards, and procedures related to the security of those systems.}&{Y}&{Y}& \\ \hline
{3.2.2 Ensure that personnel are trained to carry out their assigned information security-related duties and responsibilities.}&{N}&{Y}& \\ \hline
{3.2.3 Provide security awareness training on recognizing and reporting potential indicators of insider threat.}&{Y}&{Y}&{We would like to do more here like capture flag exercises for developers or blue/red teams events} \\ \hline
{3.3 AUDIT AND ACCOUNTABILITY}&&& \\ \hline
{3.3.1 Create and retain system audit logs and records to the extent needed to enable the monitoring, analysis, investigation, and reporting of unlawful or unauthorized system activity.}&{Y}&{Y}& \\ \hline
{3.3.2 Ensure that the actions of individual system users can be uniquely traced to those users, so they can be held accountable for their actions.}&{Y}&{Y}& \\ \hline
{3.3.3 Review and update logged events.}&{P}&{Y}&{We may look for a third party contract for this.} \\ \hline
{3.3.4 Alert in the event of an audit logging process failure.}&{N}&{Y}& \\ \hline
{3.3.5 Correlate audit record review, analysis, and reporting processes for investigation and response to indications of unlawful, unauthorized, suspicious, or unusual activity.}&{N}&{Y}&{Again shall look for third party contract for this} \\ \hline
{3.3.6 Provide audit record reduction and report generation to support on-demand analysis and reporting.}&{N}&{Y}& \\ \hline
{3.3.7 Provide a system capability that compares and synchronizes internal system clocks with an authoritative source to generate timestamps for audit records.}&{Y}&{Y}& \\ \hline
{3.3.8 Protect audit information and audit logging tools from unauthorized access, modification, and deletion.}&{Y}&{Y}& \\ \hline
{3.3.9 Limit management of audit logging functionality to a subset of privileged users.}&{Y}&{Y}& \\ \hline
{3.4 CONFIGURATION MANAGEMENT}&&& \\ \hline
{3.4.1 Establish and maintain baseline configurations and inventories of organizational systems (including hardware, software, firmware, and documentation) throughout the respective system development life cycles.}&{Y}&{Y}&{We use mainly infrastructure as code approaches so the software is well tracked. IT inventory all the hardware. } \\ \hline
{3.4.2 Establish and enforce security configuration settings for information technology products employed in organizational systems.}&{Y}&{Y}& \\ \hline
{3.4.3 Track, review, approve or disapprove, and log changes to organizational systems.}&{Y}&{Y}&{We have CCBs and code change process in place which also cover the infrastructure as code. } \\ \hline
{3.4.4 Analyze the security impact of changes prior to implementation.}&{Y}&{Y}& \\ \hline
{3.4.5 Define, document, approve, and enforce physical and logical access restrictions associated with changes to organizational systems.}&{Y}&{Y}& \\ \hline
{3.4.6 Employ the principle of least functionality by configuring organizational systems to provide only essential capabilities.}&{N}&{Y}& \\ \hline
{3.4.7 Restrict, disable, or prevent the use of nonessential programs, functions, ports, protocols, and services.}&{Y}&{Y}&{We get a lot of this by mainly containerizing the applications and having users work within deployed containers.} \\ \hline
{3.4.8 Apply deny-by-exception (blacklisting) policy to prevent the use of unauthorized software or deny-all, permit-by-exception (whitelisting) policy to allow the execution of authorized software.}&{N}&{Y}&{We need to implement SUDO lists to restrict access. However, this could be related to blacklisting of applications.} \\ \hline
{3.4.9 Control and monitor user-installed software.}&{Y}&{Y}& \\ \hline
{3.5 IDENTIFICATION AND AUTHENTICATION}&&& \\ \hline
{3.5.1 Identify system users, processes acting on behalf of users, and devices.}&{Y}&{Y}& \\ \hline
{3.5.2 Authenticate (or verify) the identities of users, processes, or devices, as a prerequisite to allowing access to organizational systems.}&{Y}&{Y}& \\ \hline
{3.5.3 Use multifactor authentication for local and network access to privileged accounts and for network access to non-privileged accounts.}&{N}&{Y}&{Chile dont require 2FA at the moment} \\ \hline
{3.5.4 Employ replay-resistant authentication mechanisms for network access to privileged and non- privileged accounts.}&{}&{Y}&{Chile dont require 2FA at the moment, but certificates are deployed to prevent mitm} \\ \hline
{3.5.5 Prevent reuse of identifiers for a defined period.}&{N}&{Y}& \\ \hline
{3.5.6 Disable identifiers after a defined period of inactivity.}&{Y}&{Y}& \\ \hline
{3.5.7 Enforce a minimum password complexity and change of characters when new passwords are created.}&{Y}&{Y}& \\ \hline
{3.5.8 Prohibit password reuse for a specified number of generations.}&{Y}&{Y}& \\ \hline
{3.5.9 Allow temporary password use for system logons with an immediate change to a permanent password.}&{Y}&{Y}& \\ \hline
{3.5.10 Store and transmit only cryptographically-protected passwords.}&{Y}&{Y}& \\ \hline
{3.5.11 Obscure feedback of authentication information.}&{Y}&{Y}& \\ \hline
{3.6 INCIDENT RESPONSE}&&& \\ \hline
{3.6.1 Establish an operational incident-handling capability for organizational systems that includes preparation, detection, analysis, containment, recovery, and user response activities.}&{Y}&{Y}&{AURA have insurance which covers this. But we really should have a contract to look over logs etc. to note when we are hit.} \\ \hline
{3.6.2 Track, document, and report incidents to designated officials and/or authorities both internal and external to the organization.}&{Y}&{Y}& \\ \hline
{3.6.3 Test the organizational incident response capability.}&{N}&{Y}& \\ \hline
{3.7 MAINTENANCE}&&& \\ \hline
{3.7.1 Perform maintenance on organizational systems.}&{Y}&{Y}& \\ \hline
{3.7.2 Provide controls on the tools, techniques, mechanisms, and personnel used to conduct system maintenance.}&{Y}&{Y}& \\ \hline
{3.7.3 Ensure equipment removed for off-site maintenance is sanitized of any CUI.}&{Y}&{Y}& \\ \hline
{3.7.4 Check media containing diagnostic and test programs for malicious code before the media are used in organizational systems.}&{Y}&{Y}& \\ \hline
{3.7.5 Require multifactor authentication to establish nonlocal maintenance sessions via external network connections and terminate such connections when nonlocal maintenance is complete.}&{N}&{Y}&{Chile dont do 2FA yet. DUO has the capability to kill sessions.} \\ \hline
{3.7.6 Supervise the maintenance activities of maintenance personnel without required access authorization.}&{Y}&{Y}& \\ \hline
{3.8 MEDIA PROTECTION}&&& \\ \hline
{3.8.1 Protect (i.e., physically control and securely store) system media containing CUI, both paper and digital.}&{N}&{Y}& \\ \hline
{3.8.2 Limit access to CUI on system media to authorized users.}&{N}&{Y}& \\ \hline
{3.8.3 Sanitize or destroy system media containing CUI before disposal or release for reuse.}&{Y}&{Y}& \\ \hline
{3.8.4 Mark media with necessary CUI markings and distribution limitations.}&{N}&{Y}&{We understand we should label rooms and machines acording to https://www.archives.gov/files/cui/20161206-cui-marking-handbook-v1-1.pdf} \\ \hline
{3.8.5 Control access to media containing CUI and maintain accountability for media during transport outside of controlled areas.}&{Y}&{Y}& \\ \hline
{3.8.6 Implement cryptographic mechanisms to protect the confidentiality of CUI stored on digital media during transport unless otherwise protected by alternative physical safeguards.}&{N}&{Y}& \\ \hline
{3.8.7 Control the use of removable media on system components.}&{}&{Y}&{Can be rolled out with puppet but there are some servers that need usb. } \\ \hline
{3.8.8 Prohibit the use of portable storage devices when such devices have no identifiable owner.}&{Y}&{Y}& \\ \hline
{3.8.9 Protect the confidentiality of backup CUI at storage locations.}&{Y}&{Y}& \\ \hline
{3.9 PERSONNEL SECURITY}&&& \\ \hline
{3.9.1 Screen individuals prior to authorizing access to organizational systems containing CUI.}&{Y}&{Y}&{Only project team members will have access to early images - all are know individuals. This doesn't suggest background security screening and it was also explicitly not required by the agencies in section 2 of the requirements document. } \\ \hline
{3.9.2 Ensure that organizational systems containing CUI are protected during and after personnel actions such as terminations and transfers.}&{Y}&{Y}& \\ \hline
{3.10 PHYSICAL PROTECTION}&&& \\ \hline
{3.10.1 Limit physical access to organizational systems, equipment, and the respective operating environments to authorized individuals.}&{Y}&{Y}&{This physical access limitations will increase with locks on server cabinets etc. but key card access is already in place. } \\ \hline
{3.10.2 Protect and monitor the physical facility and support infrastructure for organizational systems.}&{Y}&{Y}&{Security is in place on Cero Pachon and at the entrance to the mountain - though not only for Rubin so not permanently at the observatory. } \\ \hline
{3.10.3 Escort visitors and monitor visitor activity.}&{Y}&{Y}&{Actual visitors are escorted on the summit - contractors are considered more like staff. } \\ \hline
{3.10.4 Maintain audit logs of physical access.}&{N}&{Y}&{Chile use Noirlab key-card system, we should reach to them to inquire about their audit procedures} \\ \hline
{3.10.5 Control and manage physical access devices.}&{Y}&{Y}& \\ \hline
{3.10.6 Enforce safeguarding measures for CUI at alternate work sites.}&{Y}&{Y}&{This brings in \citeds{nist800-46} and \citeds{nist800-114}. Threat analysis suggested. NAT considered bad. } \\ \hline
{3.11 RISK ASSESSMENT}&&& \\ \hline
{3.11.1 Periodically assess the risk to organizational operations (including mission, functions, image, or reputation), organizational assets, and individuals, resulting from the operation of organizational systems and the associated processing, storage, or transmission of CUI.}&{Y}&{Y}& \\ \hline
{3.11.2 Scan for vulnerabilities in organizational systems and applications periodically and when new vulnerabilities affecting those systems and applications are identified.}&{N}&{Y}&{Third party contract} \\ \hline
{3.12 SECURITY ASSESSMENT}&&& \\ \hline
{3.12.1 Periodically assess the security controls in organizational systems to determine if the controls are effective in their application.}&{Y}&{Y}& \\ \hline
{3.12.2 Develop and implement plans of action designed to correct deficiencies and reduce or eliminate vulnerabilities in organizational systems.}&{Y}&{Y}& \\ \hline
{3.12.3 Monitor security controls on an ongoing basis to ensure the continued effectiveness of the controls.}&{Y}&{Y}& \\ \hline
{3.12.4 Develop, document, and periodically update system security plans that describe system boundaries, system environments of operation, how security requirements are implemented, and the relationships with or connections to other systems.}&{N}&{Y}&{Like any documentation this security documentation can get out of date. } \\ \hline
{3.13 SYSTEM AND COMMUNICATIONS PROTECTION}&&& \\ \hline
{3.13.1 Monitor, control, and protect communications (i.e., information transmitted or received by organizational systems) at the external boundaries and key internal boundaries of organizational systems.}&{Y}&{Y}& \\ \hline
{3.13.2 Employ architectural designs, software development techniques, and systems engineering principles that promote effective information security within organizational systems.}&{Y}&{Y}&{We can do more here. } \\ \hline
{3.13.3 Separate user functionality from system management functionality.}&{N}&{Y}&{This is difficult in development and commissioning but should be ok in operations. } \\ \hline
{3.13.4 Prevent unauthorized and unintended information transfer via shared system resources.}&{N}&{Y}&{This will require training the operators and scientist who have access to the CUI data to not put it on their devices. } \\ \hline
{3.13.5 Implement subnetworks for publicly accessible system components that are physically or logically separated from internal networks.}&{Y}&{Y}& \\ \hline
{3.13.6 Deny network communications traffic by default and allow network communications traffic by exception (i.e., deny all, permit by exception).}&{Y}&{Y}&{We may need to bring up iptables on each host} \\ \hline
{3.13.7 Prevent remote devices from simultaneously establishing non-remote connections with organizational systems and communicating via some other connection to resources in external networks (i.e., split tunneling).}&{Y}&{Y}& \\ \hline
{3.13.8 Implement cryptographic mechanisms to prevent unauthorized disclosure of CUI during transmission unless otherwise protected by alternative physical safeguards.}&{N}&{Y}&{IPSec and encryption coming} \\ \hline
{3.13.9 Terminate network connections associated with communications sessions at the end of the sessions or after a defined period of inactivity.}&{Y}&{Y}& \\ \hline
{3.13.10 Establish and manage cryptographic keys for cryptography employed in organizational systems.}&{Y}&{Y}& \\ \hline
{3.13.11 Employ FIPS-validated cryptography when used to protect the confidentiality of CUI.}&{N}&{Y}& \\ \hline
{3.13.12 Prohibit remote activation of collaborative computing devices and provide indication of devices in use to users present at the device.}&{Y}&{Y}&{We should take care with the new roaming camera.} \\ \hline
{3.13.13 Control and monitor the use of mobile code.}&{Y}&{Y}&{Currently we have no mobile code} \\ \hline
{3.13.14 Control and monitor the use of Voice over Internet Protocol (VoIP) technologies.}&{N}&{Y}&{Chile dont monitor voip callls} \\ \hline
{3.13.15 Protect the authenticity of communications sessions.}&{Y}&{Y}& \\ \hline
{3.13.16 Protect the confidentiality of CUI at rest.}&{N}&{Y}& \\ \hline
{3.14 SYSTEM AND INFORMATION INTEGRITY}&&& \\ \hline
{3.14.1 Identify, report, and correct system flaws in a timely manner.}&{Y}&{Y}& \\ \hline
{3.14.2 Provide protection from malicious code at designated locations within organizational systems.}&{Y}&{Y}& \\ \hline
{3.14.3 Monitor system security alerts and advisories and take action in response.}&{Y}&{Y}& \\ \hline
{3.14.4 Update malicious code protection mechanisms when new releases are available.}&{Y}&{Y}& \\ \hline
{3.14.5 Perform periodic scans of organizational systems and real-time scans of files from external sources as files are downloaded, opened, or executed.}&{Y}&{Y}& \\ \hline
{3.14.6 Monitor organizational systems, including inbound and outbound communications traffic, to detect attacks and indicators of potential attacks.}&{Y}&{Y}& \\ \hline
\textbf{Total requirements}&\textbf{}&\textbf{108}& \\ \hline
\textbf{Total Rubin Intends to comply with }&\textbf{}&\textbf{108}& \\ \hline
\textbf{Total Rubin Complies with in 2021}&\textbf{}&\textbf{72}& \\ \hline
\end{longtable} \normalsize
